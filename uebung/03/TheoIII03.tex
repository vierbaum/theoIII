\documentclass{article}
\usepackage{tikz}
\usepackage{pgfplots}
\usepackage{amsmath, amsfonts}
\usepackage{enumerate}
\usepackage{makecell}
\usepackage{xypic}
\usepackage{fancyvrb}
\input{title.tex}

\newcommand{\N}{\mathbb{N}}
\newcommand{\R}{\mathbb{R}}
\begin{document}
    \maketitle
    \section{}
    \paragraph{``$\Rightarrow$''}
    Sei $D$ ein Distanzvektor, dann gilt
    \begin{enumerate}
        \item $D_1=d(1)=0$
        \item
            Wir definieren $\pi_m$ als kürzesten Weg
            $1\rightarrow m$ für alle $m\in V\setminus\{1\}$.
            Sei $\pi_j=\pi_{j'}\circ j$
            \begin{enumerate}[i)]
                \item Ist $i=j'$, gilt
                    $d(j)=d(j')+\gamma(j',j)$
                \item Ist $i\neq j'$, so existiert
                    $\pi'_j=\pi_i\circ j$ und
                    $d(j)\leq d(i)+\gamma(i, j)$
            \end{enumerate}
        \item
            Wir nutzen die Definition von $\pi_m$ aus 2.
            Es ist also ein $j\neq 1\in V$ beliebig,
            so existiert ein $\pi_j=\pi_{i}\circ j$
            mit $d(j)=d(i)+\gamma(i, j)$
    \end{enumerate}
    \paragraph{``$\Leftarrow$''}
    Es gelten $1,2,3$, dann gilt $D_v\geq d(v)$.
    Ebenso gilt
    \begin{align*}
        D_v&\leq D_{v^n}+\gamma(v^n,v)\\
        &\leq D_{v^{n-1}}+\gamma(v^{n-1}, v^n)\gamma(v^n,v)\\
        &\vdots\\
        &\leq D_1+\gamma(1,v^1)+\hdots+\gamma(v^n,v)\\
        &=0+\gamma(1,v^1)+\hdots+\gamma(v^n,v)\\
        &=d(v)
    \end{align*}
    Es ist $d(v)\leq D_v\leq d(v)\Leftrightarrow
    D_v=d(v)\Rightarrow$ D ist ein Distanzvektor
    \section{}
    \begin{enumerate}[a)]
        \item
            Wir konstruieren mit dem Algorithmus 2.2.2 einen
            Baum $(V, E')$, wobei
            \DefineShortVerb{\|}
            \SaveVerb{VerbA}|compnum|
            \SaveVerb{VerbB}|compnum[i],compnum[j]|
            \begin{align*}
                \underset{(i,j)\in E\land (i,j)\in E}{\exists}
                &\left(\UseVerb{VerbA}^{-1}\left(
                    \min\{\UseVerb{VerbB}\}
                \right), j\right)\in E'\land\\
                &\left(\UseVerb{VerbA}^{-1}\left(
                    \max\{\UseVerb{VerbB}\}
                \right), j\right)\notin E'
            \end{align*}
            Es gilt also
            \begin{align*}
                \underset{(u,v)\in E;\notin E'}\forall
                \underset{u'\in V;(u',v)\in E'}\exists
                \UseVerb{VerbA}[u]
                \leq\UseVerb{VerbA}[u']
            \end{align*}
            Daraus folgt, ist $\phi$ für $(V,E)$ eine
            Topologische Sortierung, so auch für $(V,E')$
            Es gilt
            \begin{align*}
                \underset{(u,v)\in E'}\forall
                \UseVerb{VerbA}[v]\leq\UseVerb{VerbA}[u]
            \end{align*}
            Weiter ist also
            \begin{align*}
                \phi(v)
                =n+1-\UseVerb{VerbA}[v]
                \geq n+1-\UseVerb{VerbA}[u]=\phi(u)
            \end{align*}
    \end{enumerate}
\end{document}
