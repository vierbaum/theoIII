\documentclass{article}
\usepackage{tikz}
\usepackage{pgfplots}
\usepackage{amsmath, amsfonts}
\usepackage{enumerate}
\usepackage{makecell}
\usepackage{xypic}
\input{title.tex}

\newcommand{\N}{\mathbb{N}}
\newcommand{\R}{\mathbb{R}}
\begin{document}
	\maketitle
	\section{}
	\begin{enumerate}[a)]
		\item
            Nach Annahme gilt
            \begin{align*}
                &\limsup\limits_{n\rightarrow\infty}
                \frac{f(n)}{g(n)}<\infty\\
                &\limsup\limits_{n\rightarrow\infty}
                \frac{g(n)}{h(n)}<\infty
            \end{align*}
            Wir setzen
            \begin{align*}
                &A:=\limsup\limits_{n\rightarrow\infty}
                \frac{f(n)}{g(n)}<\infty\\
                &B:=\limsup\limits_{n\rightarrow\infty}
                \frac{g(n)}{h(n)}<\infty
            \end{align*}
            Es gilt also
            \begin{align*}
                \lim\limits_{n\rightarrow\infty}
                \sup\frac{f(n)}{g(n)}\cdot\sup\frac{g(n)}{h(n)}
                &=A\cdot B\\
                \Leftrightarrow\lim\limits_{n\rightarrow\infty}
                \sup\left(\frac{f(n)}{g(n)}
                \cdot\frac{g(n)}{h(n)}\right)
                &\leq A \cdot B\\
                \Leftrightarrow\lim\limits_{n\rightarrow\infty}
                \sup\left(\frac{f(n)}{h(n)}\right)
                \leq A\cdot B&<\infty\\
                \Leftrightarrow\limsup\limits_{n\rightarrow\infty}
                \frac{f(n)}{h(n)}&<\infty\\
                &\Leftrightarrow
                f\in
                O(h)
            \end{align*}
        \item
            Nach Annahme gilt
            \begin{align*}
                A:=\limsup\limits_{n\rightarrow\infty}
                \frac{f_1(n)}{g_1(n)}<\infty\\
                B:=\limsup\limits_{n\rightarrow\infty}
                \frac{f_2(n)}{g_2(n)}<\infty
            \end{align*}
            \begin{align*}
                \infty&>A\cdot B\\
                &=\limsup\limits_{n\rightarrow\infty}\frac{f_1}{g_1}\cdot
                \limsup\limits_{n\rightarrow\infty}\frac{f_2}{g_2}\\
                &\geq
                \limsup\limits_{n\rightarrow\infty}
                \left(\frac{f_1}{g_1}\cdot\frac{f_2}{g_2}\right)\\
                &=\limsup\limits_{n\rightarrow\infty}
                \left(\frac{f_1\cdot f_2}{g_1\cdot g_2}\right)\\
                &\Leftrightarrow (f_1\cdot f_2)\in O(g_1\cdot g_2)
            \end{align*}
    \end{enumerate}
    \section{}
    \begin{enumerate}[a)]
        \item
            Da $a^n$ und $n^k$ Abbildungen von $\N$ sind und demnach nicht stetig
            sind, können wir Die Regel von L'Hospital nicht verwenden.
            Demnach definieren wir zunächst zwei funktionen\\
            $\phi: \R^+\rightarrow \R, x\mapsto x^k$ für $k\in\R$ fest.\\
            Weiter formen wir $\phi(x)=x^{m+\alpha}$ mit $m\in\N_0$ und $\alpha\in[0,1)$.\\
            Sowie $\psi: \R^+\rightarrow \R, x\mapsto a^x$ für $a>1$ fest\\
            Hierbei gilt $\underset{n\in\N}\forall \frac{n^k}{a^n}=\frac{\phi(n)}{\psi(n)}$.
            Da $\lim\limits_{x\rightarrow\infty} \phi(x)=\infty$ und
            $\lim\limits_{x\rightarrow\infty} \psi(x)=\infty$, und beide auf $(0,\infty)$ stetig diff.bar sind,
            können wir die Regel von L'Hospital anwenden. Also gilt
            \begin{align*}
                \lim\limits_{x\rightarrow\infty}\frac{\phi(x)}{\psi(x)}
                =\lim\limits_{x\rightarrow\infty}\frac{\alpha\cdot m!\cdot x^{\alpha-1}}{\log^{m+1}(a)\cdot a^x}=0
            \end{align*}
            Da $\alpha -1<0$ geht $x^{\alpha-1}$ gegen 0.
            Nach cauchy gilt demnach
            \begin{equation*}
                \underset{\varepsilon>0}\forall\,\underset{n_0\in\R}\exists\,
                \underset{n,m\in\R;n,m\geq n_0}\forall
                \left|
                    \frac{\phi(n)}{\psi(n)}-\frac{\phi(m)}{\psi(m)}
                \right|<\varepsilon
            \end{equation*}
            Da $n,m\in\R$ beliebig, formulieren wir um
            \begin{equation*}
                \underset{\varepsilon>0}\forall\,\underset{n_0\in\R}\exists\,
                \underset{n,m\in\N;n,m\geq n_0}\forall
                \left|
                    \frac{\phi(n)}{\psi(n)}-\frac{\phi(m)}{\psi(m)}
                \right|<\varepsilon
            \end{equation*}
            Demnach gilt
            \begin{equation*}
                \lim\limits_{n\rightarrow\infty}\frac{\phi(n)}{\psi(n)}=
                \lim\limits_{n\rightarrow\infty}\frac{n^k}{a^n}=0
            \end{equation*}
            Es folgt also $n^k\in o(a^n)$
        \item
            Ähnlich wie bei a) definieren wir auch wieder funktionen\\
            $\phi: \R^+\rightarrow \R, x\mapsto \log^r(x)$ für $r\in\R$ fest.\\
            Weiter formen wir $\phi(x)=log^{m+\alpha}(x)$ mit $m\in\N_0$ und $\alpha\in[0,1)$.\\
            Sowie $\psi: \R^+\rightarrow \R, n^l$ für $a>1$ fest.\\
            Es gilt also
            \begin{align*}
                \lim\limits_{x\rightarrow\infty}
                \frac{\phi}{\psi}
                &=
                \lim\limits_{x\rightarrow\infty}
                \frac{\log^{m+\alpha}(x)}{x^l}\\
                &=
                \lim\limits_{x\rightarrow\infty}
                \frac{m!\log^{\alpha}(x)}{l^m\cdot x^l}\\
                &=
                \lim\limits_{x\rightarrow\infty}
                \frac{\alpha\cdot m!\log^{\alpha-1}(x)}{l^{m+1}\cdot x^l}=0\\
            \end{align*}
            Ebenfalls können wir wie bei a demnach zeigen, dass mittels cauchy
            auch die folge $\left(\frac{\phi(n)}{\psi(n)}\right)_{n\in\N}$
            gegen 0 konvergiert. Demnach gilt
            $\log^r(n)\in o(n^l)$
    \end{enumerate}
    \section{}
    \begin{enumerate}[a)]
        \item
            Induktionsvorraussetzung:
            \begin{equation*}
                \sum_{i=1}^1 i=1=\frac{1\cdot(1+1)}{2}
            \end{equation*}
            Induktionsschritt:
            \begin{align*}
                &\sum_{i=1}^{n+1} i\\
                =&\sum_{i=1}^{n} i + n + 1\\
                =&\frac{n(n+1)}{2} + n + 1\\
                =&\frac{n(n+1)}{2} + \frac{2(n+1)}{2}\\
                =&\frac{(n+1)((n+1)+1)}{2}\\
            \end{align*}
        \item
            $n^{0.4}<\sqrt{n}<n\log n<3n<3^\frac{n}{2}<2^n<n!$
    \end{enumerate}
    \section{}
    \begin{enumerate}[f1)]
        \item
            $\Theta(n^2)$
            \begin{tabular}{cc|cccc}
                &j&1&2&$\hdots$&n\\
                i&\\
                \hline
                1&&$\times$&$\times$&$\hdots$&$\times$\\
                2&&$\times$&$\times$&$\hdots$&$\times$\\
                $\vdots$&\\
                n&&$\times$&$\times$&$\hdots$&$\times$
            \end{tabular}
        \item
            $\Theta(n^2)$
            \begin{tabular}{cc|cccc}
                &j&1&2&$\hdots$&n\\
                i&\\
                \hline
                1&&$\times$&$\times$&$\hdots$&$\times$\\
                2&&&$\times$&$\hdots$&$\times$\\
                $\vdots$&&&&$\ddots$&$\vdots$\\
                n&&&&&$\times$
            \end{tabular}
        \item
            $\Theta(n^2)$
            \begin{tabular}{cc|cccc}
                &j&1&2&$\hdots$&n\\
                i&\\
                \hline
                1&&$\times$&\\
                2&&$\times$&$\times$\\
                $\vdots$&&$\vdots$&&$\ddots$\\
                n&&$\times$&$\times$&$\hdots$&$\times$
            \end{tabular}
        \item
            $\Theta(n)$
            \begin{tabular}{cc|cccc}
                &j&1&2&$\hdots$&1000\\
                i&\\
                \hline
                1&&$\times$&$\times$&$\hdots$&$\times$\\
                2&&$\times$&$\times$&$\hdots$&$\times$\\
                $\vdots$&\\
                1000n&&$\times$&$\times$&$\hdots$&$\times$
            \end{tabular}
        \item
            $\Theta(2^n)$
            Genau $2^{n-1}$ Aufrufe:
            IA:
            \begin{align*}
                |f5(1)|=1=2^{1-1}
            \end{align*}
            IS:
            \begin{align*}
                |f5(n+1)|=|f5(n)|+|f5(n)|=2^{n-1}+2^{n-1}=2^{(n-1)+1}
            \end{align*}
        \item
            $\Theta(2^n)$\\
            \xymatrix{
                &&&&&&f(8)\ar[dlll]\ar[drrr]\\
                &&&f(4)\ar[dll]\ar[dr]&&&&&&f(4)\ar[dl]\ar[drr]\\
                &f(2)\ar[dl]\ar[dr]&&&f(2)\ar[dl]\ar[dr]&&&&f(2)\ar[dl]\ar[dr]&&&f(2)\ar[dl]\ar[dr]\\
                f(1)&&f(1)&f(1)&&f(1)&&f(1)&&f(1)&f(1)&&f(1)
            }
        \item
            $\Theta(\log(n)$
            \xymatrix{
                f(8)\ar[r]&f(4)\ar[r]&f(2)\ar[r]&f(1)
            }
        \item
            $\Theta(\log(n)$
            \xymatrix{
                f(8)\ar[r]&f(4)\ar[r]&f(2)\ar[r]&f(1)
            }
        \item
            $\Theta(n^2)$
            \begin{tabular}{c|ccccc}
                &f(0)&f(1)&f(2)&f(3)&f(4)\\
                \hline
                f(0)&1&2&3&4&5\\
                f(1)&&1&2&3&4\\
                f(2)&&&1&2&3\\
                f(3)&&&&1&2\\
                f(4)&&&&&1\\
            \end{tabular}
        \item
            $\Theta(1)$
        \item
            $\Theta(\log(n))$
            \xymatrix{
                f(8)\ar[r]&f(4)\ar[r]&f(2)\ar[r]&f(1)
            }
        \item
            $\Theta(\log\log(n))$
    \end{enumerate}
\end{document}
