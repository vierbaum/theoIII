\documentclass{article}
\usepackage{tikz}
\usepackage{pgfplots}
\usepackage{amsmath, amsfonts}
\usepackage{enumerate}
\usepackage{makecell}
\usepackage{xypic}
\input{title.tex}

\newcommand{\N}{\mathbb{N}}
\newcommand{\R}{\mathbb{R}}
\begin{document}
	\maketitle
	\section{}
	\begin{enumerate}[a)]
		\item
		\item
		    Wir wissen, dass $\log x$ auf $x>0$ monoton
			steigt.
			Demnach gilt
			\begin{equation*}
				\sum_{i=1}^n \log i\leq
				\sum_{i=1}^n \log n=n\log n
            \end{equation*}
            Weiter ist
            \begin{align*}
                \limsup\limits_{n\in\N^+}
                \frac{\sum_{i=1}^n\log i}{n\log n}
                \leq
                \limsup\limits_{n\in\N^+}
                \frac{n\log n}{n\log n}
                =1<\infty
            \end{align*}
            Es gilt
            \begin{equation*}
                \sum_{i=1}^n \log i\in O(n\log n)
            \end{equation*}

            Es gilt $0<\log k<\log n(1<k<n)$ und demnach
            gilt
            \begin{equation*}
                \underset{k<n\in\N^+}\forall\,
                \underset{a>1}\exists
                a\log(k)=\log(n)
            \end{equation*}
            Wir definieren $a_k$ mit
            \begin{equation*}
                a_k = \frac{\log n}{\log k}.
            \end{equation*}
            Somit gilt
            \begin{align*}
                \liminf\limits_{n\in\N^+}
                \frac{\sum_{i=1}^n\log i}{n\log n}
                &=
                \liminf\limits_{n\in\N^+}
                \frac{1\cdot\log n+a_{n-1}\log n+\hdots+a_1\log n}{\log n}\\
                &=
                \liminf\limits_{n\in\N^+}
                \frac{\log n(1+a_{n-1}+\hdots+a_1)}{n\log n}\\
                &=
                \liminf\limits_{n\in\N^+}
                \frac{1+a_{n-1}+\hdots+a_1}{n}\\
                &\geq
                \liminf\limits_{n\in\N^+}
                \frac{\sum_{k=1}^n 1}{n}\\
                &=1>0
            \end{align*}
            Demnach gilt $\sum_{i=1}^n\log i\in\Omega(n\log n)$
            und $\sum_{i=1}^n\log i\in\Theta(n\log n)$
    \end{enumerate}
    \section{}
    \xymatrix{
        &&&&1\ar[dll]\ar[drr]\\
        &&2\ar[dl]\ar[dr]&&&&3\ar[dl]\ar[dr]\\
        &4\ar[dl]\ar[dr]&&5&&6&&7\\
        \vdots&&\ddots
    }
    Dann gilt für knoten $n$, die linke kante $(n,2n)$
    und die rechte $(n,2n+1)$
    und für beliebigen knoten $m$ der
    $k$ mal rechts von $n$ liegt ist
    $m=2^k+\sum_{i=0}^{k-1}2^i$
    Somit ist
    Also ist für beliebigen knoten $n$ links-rechts-$\hdots$
    $=A_n=\{(n,2n),(2n\cdot \left[2^k+\sum_{i=0}^{k-1}2^i\right],2n\cdot \left[2^{k+1}+\sum_{i=0}^{k}2^i)\right]|k>1\}$
    Wir nehmen an, es existieren $n\neq m$ mit
    $A_n\cap A_m\neq\emptyset$.
    Dann gilt
    \begin{align*}
    	&\{(n,2n),(2n\cdot \left[2^k+\sum_{i=0}^{k-1}2^i\right],2n\cdot \left[2^{k+1}+\sum_{i=0}^{k}2^i)\right]|k>1\}\\
    	\cap&\{(m,2m),(2m\cdot \left[2^k+\sum_{i=0}^{k-1}2^i\right],2m\cdot \left[2^{k+1}+\sum_{i=0}^{k}2^i)\right]|k>1\}\neq\emptyset\\
        \Leftrightarrow&\underset{j,k>1}\exists:
        2n\cdot \left[2^k+\sum_{i=0}^{k-1}2^i\right]
        =2m\cdot \left[2^j+\sum_{i=0}^{j-1}2^i\right]\\
        \Leftrightarrow&\underset{j,k>1}\exists:
        \frac{n}{m}\cdot \left[2^k+\sum_{i=0}^{k-1}2^i\right]
        =\left[2^j+\sum_{i=0}^{j-1}2^i\right]\tag 1\\
        \Leftrightarrow&\underset{j,k>1}\exists:
        \left[2^k+\sum_{i=0}^{k-1}2^i\right]
        =\frac{m}{n}\cdot \left[2^j+\sum_{i=0}^{j-1}2^i\right]\tag 2\\
    \end{align*}
    Da die summen ganzzahlig sind, folgt aus (1) $m$ tilt
    $n$ und aus $2$ folgt $n$ teilt $m$, also
    ist $n=m$, das ist ein widerspruch zur annahme

    Ist bspw. links-links-rechts-rechts-$hdots$
    ist $A_1\cap A_2=\{(2,4)\}$
    \section{}
    \section{}
    \paragraph{Reflexivität}
    Da der Pfad von $a$ nach $a$ existiert
    ist $aRa$
    \paragraph{Symmetrie}
    ist $aRb$ existiert ein pfad von $a$ nach $b$
    und von $b$ nach $a$
    also gilt  $bRa$
    \paragraph{Transitivität}
    Seien $a,b,c$ 3 Knoten mit
    $aRb$ und $bRc$.
    Also existieren ein Pfade
    $\pi_0=a\hdots b,\pi_1=b\hdots c$.
    Demnach existieren Pfade $\pi_3=\pi_0\pi_1=a\hdots c,
    \pi_3'=c\hdots a$ und es ist $aRc$ also
    ist $R$ eine Äquivalenzrelation.

    ist $a\rightarrow b$, so gilt
    $\underset{u\in[a]}u\rightarrow b$ da $u\rightarrow a\rightarrow b$
    ist nun $a\rightarrow b\rightarrow c$, gilt
    $\underset{u\in[a]\cup[b]}u\rightarrow b$
    und ist $a\rightarrow b\rightarrow c\rightarrow a$, gilt
    $\underset{u\in[a]\cup[b]\cup[c]}u\rightarrow a\land a\rightarrow u$,
    denn ist
    \begin{enumerate}
        \item $u\in[a]\Leftrightarrow a\sim u$,
        \item $u\in[b]\Leftrightarrow a\rightarrow b\rightarrow u(u\sim b)$
        \item $u\in[c]\Leftrightarrow a\rightarrow b\rightarrow c\rightarrow u(u\sim c)$
    \end{enumerate}
    Demnach gilt $u\sim a$, $a\sim b\sim c$ und $[a]=[b]=[c]$
\end{document}
